\documentclass[12pt, a4paper]{article}
\usepackage[utf8]{inputenc}
%\usepackage[IL2]{fontenc}
\usepackage[czech]{babel}
\usepackage[pdftex]{graphicx}
\usepackage{mathtools}
\usepackage{amsmath}
\usepackage{svg}
\usepackage{textcomp}
\usepackage{listings,xcolor}
\usepackage[final]{pdfpages}
\usepackage{verbatim}
\usepackage{fancyhdr}
\usepackage[T1]{fontenc}

\usepackage[nottoc,notlot,notlof]{tocbibind}
\usepackage[pdftex,hypertexnames=false]{hyperref}
\hypersetup{colorlinks=true,
  unicode=true,
  linkcolor=black,
  citecolor=black,
  urlcolor=black,
  bookmarksopen=true}

\title{\textbf{Dokumentace semestrální práce} \\KIV/NET}
\author{Vojtěch Danišík}
\begin{document}

\begin{titlepage} 
	\newcommand{\HRule}{\rule{\linewidth}{0.5mm}} 
	\begin{center}
	\includegraphics[width=12cm]{img/fav_logo}\\
	\end{center}
	\textsc{\LARGE Západočeská univerzita v Plzni}\\[1.5cm] 	
	\textsc{\Large Programování v prostředí .NET}\\[0.5cm] 
	\textsc{\large KIV/NET}\\[0.5cm] 
	\HRule\\[0.4cm]
	{\huge\bfseries Zpracování naměřených teplot}\\[0.4cm] 
	\HRule\\[1.5cm]
	
	\begin{minipage}{0.4\textwidth}
		\begin{flushleft}
			\large
			Vojtěch \textsc{Danišík}\newline
			A19N0028P\newline
			danisik@students.zcu.cz
		\end{flushleft}
	\end{minipage}
	\vfill\vfill\vfill
	\begin{flushright}
	{\large\today}
	\end{flushright}
	\vfill 
\end{titlepage}

\newpage
Souhlasím s vystavením této semestrální práce na stránkách katedry \newline informatiky a výpočetní techniky a jejímu využití pro prezentaci pracoviště.
\newline
\begin{flushright}
Vojtěch Danišík
\end{flushright}

\newpage
\tableofcontents

\newpage
\section{Zadání}
Desktopová aplikace v programovacím jazyce C\#, která umožňuje zaznamenat naměřené teploty ve městech za daný měsíc. Cílem aplikace je možnost zaznamenat si naměřené teploty za daný měsíc pro určité město v daný rok. Zaznamenané teploty se poté zprůměrují v daný měsíc a lze si následně vykreslit 3 grafy:
\begin{itemize}
\item Průměrná teplota ve všech městech po měsíci v daný rok.
\item Porovnání teplot dvou měst po měsících v daný rok.
\item Teplota ve městech za měsíc v daný rok.
\end{itemize}
V aplikaci si uživatel bude moct zvolit jednotku teploty. Zvolená jednotka teploty bude aplikována na data v aktuálně zpracovávaném roce. Na výběr budou 3 jednotky:
\begin{itemize}
\item Stupeň Fahrenheita [°F]
\item Stupeň Celsia [°C]
\item Kelvin [K]
\end{itemize}
Aplikace bude umožňovat přidávání / mazání měst, přidávání / mazání dat z celého roku. Aplikace dále umožní zaznamenané teploty v daný rok exportovat do csv.

\section{Analýza}
TODO
\newline
Popište, jakou architekturu jste vybrali. Popište á pomocí vhodných diágrámů zobrázte, ják vypádájí jednotlivé části (nápř. ER model pro záchycení dátové vrstvy, átp.)

\section{Implementace}
TODO
\newline
Uveďte použité technologie v jednotlivých vrstvách, á jákým způsobem jsou v aplikaci použity.

\section{Uživatelská příručka}
TODO
\newline
Nápište stručnou uživátelskou příručku (Bodnik, 2013) k výsledné aplikaci. Pokud program obsahuje předdefinováné uživátele, uveďte jméná á heslá, pod kterými je možné se přihlásit á otestovát funkčnost.

\section{Závěr}
TODO
\newline
Stručně zhodnoťte výslednou áplikáci á problémy, se kterými jste se během řešení potkáli.

\section{Reference}
TODO
\newline


\setcounter{section}{0}
\renewcommand{\thesection}{\Alph{section}}
\section{Programátorský deník}
TODO
\newline
Evidence áktivit by Vám mohlá v budoucnosti pomoct s odhadem čásu, který je potřebá pro řešení dáného problému. Ve vlástním zájmu tedy uvádějte reálné čásy – z pohledu předmětu .NET nádsázením nebo podhodnocením čásů nic nezískáte. Stáčí nějáký jednoduchý formát, nápř. tabulka: Datum | Aktivita | Délka [hod].

\end{document}